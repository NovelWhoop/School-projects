% 2. projekt do predmetu ITY
% autor: Roman Halik
% login: xhalik01

\documentclass[11pt, a4paper]{article}

\usepackage[left=1.5cm,text={18cm, 25cm},top=2.5cm]{geometry}
\usepackage[czech]{babel}
\usepackage{times}
\usepackage[latin2]{inputenc}
\usepackage[IL2]{fontenc}
\usepackage{amsmath}
\usepackage{amsthm}
\usepackage{amssymb}

\theoremstyle{plain}
\newtheorem{noreset}{DEFAULT}[section]
\theoremstyle{definition}
\newtheorem{definice}[noreset]{Definice}
\theoremstyle{plain}
\newtheorem{algoritmus}[noreset]{Algoritmus}
\newtheorem{veta}{Vìta}
  
\begin {document}

\thispagestyle{empty} % vypne èíslování stránek
\setcounter{page}{0} % na dalsi strane se zacne od 1

\begin{center}
  \Huge
  \textsc{Fakulta informaèních technologií\\[-4mm]Vysoké uèení technické v~Brnì\\}
  \vspace{\stretch{0.382}}
  \LARGE Typografie a~publikování\,--\,2.\,projekt\\[-2mm]Sazba dokumentù a~matematických výrazù
  \vspace{\stretch{0.618}}
\end{center}
{\Large 2015 \hfill Roman Halík}

\newpage
\twocolumn

\section*{Úvod} \label{sec:uvod}
  V~této úloze si vyzkou¹íme sazbu titulní strany, matematických vzorcù, prostøedí a~dal¹ích textových struktur obvyklých pro technicky zamìøené texty (napøíklad rovnice \eqref{eq:1} nebo definice \ref{def:1} na stranì \pageref{sec:uvod}).

  Na titulní stranì je vyu¾ito sázení nadpisu podle optického støedu s~vyu¾itím zlatého øezu. Tento postup byl probírán na pøedná¹ce.

  \section {Matematický text}
  Nejprve se podíváme na sázení matematických symbolù a~výrazù v~plynulém textu. Pro mno¾inu $V$ oznaèuje $\mathrm{card}(V)$ kardinalitu $V$.
  Pro mno¾inu $V$ reprezentuje $V^*$ volný monoid generovaný mno¾inou $V$ s~operací konkatenace.
  Prvek identity ve volném monoidu $V^*$ znaèíme symbolem $\varepsilon$.
  Nech» $V^+ = V^* - \{ \varepsilon \}$. Algebraicky je tedy $V^+$ volná pologrupa generovaná mno¾inou $V$ s~operací konkatenace.
  Koneènou neprázdnou mno¾inu $V$ nazvìme \emph{abeceda. }
  Pro $w \in V^*$ oznaèuje $|w|$ délku øetìzce $w$. Pro $W \subseteq V$ oznaèuje $\mathrm{occur}(w,W)$ poèet výskytù symbolù z~$W$ v~øetìzci $w$ a~$\mathrm{sym}(w,i)$ urèuje $i$-tý symbol øetìzce $w;$ napøíklad $\mathrm{sym}(abcd,3) = c$.
  
  Nyní zkusíme sazbu definic a~vìt s~vyu¾itím balíku {\fontfamily{pcr}\selectfont amsthm}.


  \begin{definice} 
    \label{def:1}\emph{Bezkontextová gramatika} je ètveøice $G = (V,T,P,S),$ kde $V$ je totální abeceda, $T \subseteq V$ je abeceda terminálù, $S \in (V-T)$ je startující symbol a~$P$ je koneèná mno¾ina pravidel tvaru $q\colon A \rightarrow \alpha$,kde $A \in (V-T), \alpha \in V^*$ a~$q$ je návì¹tí tohoto pravidla. Nech» $N=V-T$ znaèí abecedu neterminálù. Pokud $q\colon A \rightarrow \alpha \in P, \gamma, \delta \in V^*, G$ provádí derivaèní krok z~$\gamma A\delta$ do $\gamma \alpha \delta$ podle pravidla $q\colon A \rightarrow \alpha$, symbolicky pí¹eme $\gamma A \delta \Rightarrow \gamma \alpha \delta  [ q\colon A \rightarrow \alpha ]$ nebo zjednodu¹enì $\gamma A \delta \Rightarrow \gamma \alpha \delta$. Standardním zpùsobem definujeme $\Rightarrow^m$, kde $m \geq 0$. Dále definujeme tranzitivní uzávìr $\Rightarrow^+$ a~tranzitivnì-reflexivní uzávìr $\Rightarrow^*$. 
  \end{definice}

  Algoritmus mù¾eme uvádìt podobnì jako definice textovì, nebo vyu¾ít pseudokódu vysázeného ve vhodném prostøedí (napøíklad {\fontfamily{pcr}\selectfont algorithm2e}).

  \begin{algoritmus}
    Algoritmus pro ovìøení bezkontextovosti gramatiky. Mìjme gramatiku G = (N, T, P, S).
    \begin{enumerate}
      \item \label{item:1}Pro ka¾dé pravidlo $p \in P$ proveï test, zda $p$ na levé stranì obsahuje právì jeden symbol z~$N$.
      \item Pokud v¹echna pravidla splòují podmínku z~kroku \ref{item:1}, tak je gramatika $G$ bezkontextová.
    \end{enumerate}
  \end{algoritmus}

  \begin{definice}
    Jazyk definovaný gramatikou $G$ definujeme jako $L(G) = \{ w \in T^*|S \Rightarrow^* w \}$.
  \end{definice}

    \subsection {Podsekce obsahující vìtu}

      \begin{definice}
        Nech» $L$ je libovolný jazyk. $L$ \emph{je bezkontextový jazyk }, kdy¾ a~jen kdy¾ $L = L(G)$ kde $G$ je libovolná bezkontextová gramatika.
      \end{definice}

      \begin{definice}
        Mno¾inu $\mathcal{L}_{CF} = \{ L|L$ je bezkontextový jazyk$\}$ nazýváme \emph{tøídou bezkontextových jazykù. }
      \end{definice}

      \begin{veta}
        \label{veta:1}
        Nech» $L_{abc} = \{ a^n b^n c^n | n \geq 0 \}$. Platí, ¾e $L_{abc} \not\in \mathcal{L}_{CF}$.
      \end{veta}

      \begin{proof}
        Dùkaz se provede pomocí Pumping lemma pro bezkontextové jazyky, kdy uká¾eme, ¾e není mo¾né, aby platilo, co¾ bude implikovat pravdivost vìty \ref{veta:1}.
      \end{proof}

  \section {Rovnice a odkazy}
    Slo¾itìj¹í matematické formulace sázíme mimo plynulý text. Lze umístit nìkolik výrazù na jeden øádek, ale pak je tøeba tyto vhodnì oddìlit, napøíklad pøíkazem {\verb \quad }.

    \begin{equation*}
      \sqrt[x^2]{y^3_0} \quad \mathbb{N} = {\{0,1,2,\ldots \} \quad x^{y^y} \neq x^{yy} \quad z_{i_j} \not\equiv z_{ij}}
    \end{equation*}

    V~rovnici \eqref{eq:1} jsou vyu¾ity tøi typy závorek s~rùznou explicitnì definovanou velikostí.

    \begin{eqnarray}  
      \label{eq:1}
      \bigg\{ \Big[ \big(a + b \big) *c \Big]^d + 1 \bigg\} & = & x \\
      \nonumber\lim_{x \rightarrow \infty} \frac{\sin^2 x + \cos^2 x}{4} & = & y
    \end{eqnarray}

    V~této vìtì vidíme, jak vypadá implicitní vysázení limity $\lim_{n \rightarrow \infty} f(n)$ v~normálním odstavci textu. Podobnì je to i~s~dal¹ími symboly jako $\sum^n_1$ èi $\bigcup_{A \in \mathcal{B}}$. V pøípadì vzorce $\lim\limits_{x \rightarrow 0} \frac{\sin x}{x} = 1$ jsme si vynutili ménì úspornou sazbu pøíkazem {\verb \limits }.

    \begin{eqnarray}    
      \int\limits_a^b f(x) \, \mathrm{d}x & = & - \int_b^a f(x) \, \mathrm{d}x \\
      \left(\sqrt[5]{x^4} \right)' = \left(x^{\frac{4}{5}}\right)' & = & \frac{4}{5}x^{-\frac{1}{5}} = \frac{4}{5\sqrt[5]{x}} \\
      \overline{\overline{A \vee B}} & = & \overline{\overline{A} \wedge \overline{B}}
    \end{eqnarray}

  \section {Matice}
    Pro sázení matic se velmi èasto pou¾ívá prostøedí array a~závorky ({\verb \left }, {\verb \right }).

    $$\left( \begin{array}{cc}
    a+b & b-a \\ 
    \widehat{\xi+\omega} & \hat{\pi} \\
    \vec{a} & \overleftrightarrow{AC} \\
    0 & \beta
    \end{array}\right)$$

    $$A=
    \left\| \begin{array}{cccc}
    a_{11} & a_{12} & \hdots & a_{1n} \\
    a_{21} & a_{22} & \hdots & a_{2n} \\
    \vdots & \vdots & \ddots & \vdots \\
    a_{m1} & a_{m12} & \hdots & a_{mn}
    \end{array} \right\|$$

    $$\left| \begin{array}{cc}
    t & u~\\
    v~& w
    \end{array} \right|
    = tw-uv$$

    Prostøedí {\fontfamily{pcr}\selectfont array} lze úspì¹nì vyu¾ít i~jinde.

    $$\binom{n}{k} = \left\{ \begin{array}{ll}
    \frac{n!}{k!(n-k)!} & \text{pro } 0 \leq k~\leq n\\
    0 & \text{pro }k < 0 \text{ nebo } k~> n
    \end{array}\right.$$

  \section {Závìrem}
    V~pøípadì, ¾e budete potøebovat vyjádøit matematickou konstrukci nebo symbol a~nebude se Vám daøit jej nalézt v~samotném \LaTeX u, doporuèuji prostudovat mo¾nosti balíku maker \AmS-\LaTeX .
    Analogická pouèka platí obecnì pro jakoukoli konstrukci v~\TeX u.

\end {document}
